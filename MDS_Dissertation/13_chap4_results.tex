\chapter{Results and Discussion} 

%\section{Findings}
Lorem ipsum dolor sit amet, consectetur adipiscing elit, sed do eiusmod tempor incididunt ut labore et dolore magna aliqua. Ut enim ad minim veniam, quis nostrud exercitation ullamco laboris nisi ut aliquip ex ea commodo consequat. Duis aute irure dolor in reprehenderit in voluptate velit esse cillum dolore eu fugiat nulla pariatur. Excepteur sint occaecat cupidatat non proident, sunt in culpa qui officia deserunt mollit anim id est laborum.

\section{Experimental Setup}


\section{Model Performance}

\subsection{Loss Curves Comparison}

Figures~\ref{fig:rawparadata-loss} illustrate training and validation loss curves. The RawData model exhibits rapid convergence and stabilizes at a significantly lower loss ($0.0902$) compared to the ParaData model ($0.4254$). This suggests better adaptation and generalization capabilities when fine-tuned directly on the original legal dataset.

\begin{figure}[H]
    \centering
    \includegraphics[width=\textwidth]{figures/Train-Eval_LossCurve.png}
    \captionsetup{labelfont={bf,it}, textfont={bf,it}} % <- Makes only this caption bold italic
    \caption{Training and validation loss observed during training of Raw and Paraphrased Data Model}
    \label{fig:rawparadata-loss}
\end{figure}

Lorem ipsum dolor sit amet, consectetur adipiscing elit, sed do eiusmod tempor incididunt ut labore et dolore magna aliqua. Ut enim ad minim veniam, quis nostrud exercitation ullamco laboris nisi ut aliquip ex ea commodo consequat. Duis aute irure dolor in reprehenderit in voluptate velit esse cillum dolore eu fugiat nulla pariatur. Excepteur sint occaecat cupidatat non proident, sunt in culpa qui officia deserunt mollit anim id est laborum.


\subsection{Training Speed and Runtime Analysis}

Analysis of runtime and evaluation speed metrics further provides insights into training dynamics. Curves such as \texttt{eval/runtime}, \texttt{eval/steps\_per\_second}, and\\ \texttt{eval/samples\_per\_second} (illustrated in Figure \ref{fig:rawparadata-runtime}) indicate consistent evaluation efficiency. Specifically, the RawData model maintains a stable runtime with higher evaluation throughput, suggesting slightly more efficient utilization of computational resources.

\begin{figure}[H]
    \centering
    \includegraphics[width=\textwidth]{figures/TrainingSpeedandRuntimeAnalysis.png}
    \captionsetup{labelfont={bf,it}, textfont={bf,it}} % <- Makes only this caption bold italic
    \caption{Evaluation runtime and throughput metrics for Raw and Paraphrased Data Model}
    \label{fig:rawparadata-runtime}
\end{figure}


\subsection{Training Dynamics and Hyperparameter Trends}

Lorem ipsum dolor sit amet, consectetur adipiscing elit, sed do eiusmod tempor incididunt ut labore et dolore magna aliqua. Ut enim ad minim veniam, quis nostrud exercitation ullamco laboris nisi ut aliquip ex ea commodo consequat. Duis aute irure dolor in reprehenderit in voluptate velit esse cillum dolore eu fugiat nulla pariatur. Excepteur sint occaecat cupidatat non proident, sunt in culpa qui officia deserunt mollit anim id est laborum.

\begin{figure}[H]
    \centering
    \includegraphics[width=150mm]{figures/Raw - Training Dynamics and Hyperparameter Trends.png}
    \captionsetup{labelfont={bf,it}, textfont={bf,it}} % <- Makes only this caption bold italic
    \caption{Training dynamics (learning rate, gradient norms, steps, and epochs) for Raw Data Model.}
    \label{fig:rawdata-training-dynamics}
\end{figure}

\begin{figure}[htbp]
    \centering
    \includegraphics[width=150mm]{figures/Para - Training Dynamics and Hyperparameter Trends.png} 
    \captionsetup{labelfont={bf,it}, textfont={bf,it}} % <- Makes only this caption bold italic
    \caption{Training dynamics (learning rate, gradient norms, steps, and epochs) for Paraphrased Data Model.}
    \label{fig:paradata-training-dynamics}
\end{figure}


\newpage
\section{Model Inference}


\section{Evaluation Results}

\section{Qualitative Observations}

Lorem ipsum dolor sit amet, consectetur adipiscing elit, sed do eiusmod tempor incididunt ut labore et dolore magna aliqua. Ut enim ad minim veniam, quis nostrud exercitation ullamco laboris nisi ut aliquip ex ea commodo consequat. Duis aute irure dolor in reprehenderit in voluptate velit esse cillum dolore eu fugiat nulla pariatur. Excepteur sint occaecat cupidatat non proident, sunt in culpa qui officia deserunt mollit anim id est laborum.

\section{Discussion and Interpretation}
Lorem ipsum dolor sit amet, consectetur adipiscing elit, sed do eiusmod tempor incididunt ut labore et dolore magna aliqua. Ut enim ad minim veniam, quis nostrud exercitation ullamco laboris nisi ut aliquip ex ea commodo consequat. Duis aute irure dolor in reprehenderit in voluptate velit esse cillum dolore eu fugiat nulla pariatur. Excepteur sint occaecat cupidatat non proident, sunt in culpa qui officia deserunt mollit anim id est laborum.




\section{Comparison with Existing Approaches}
Lorem ipsum dolor sit amet, consectetur adipiscing elit, sed do eiusmod tempor incididunt ut labore et dolore magna aliqua. Ut enim ad minim veniam, quis nostrud exercitation ullamco laboris nisi ut aliquip ex ea commodo consequat. Duis aute irure dolor in reprehenderit in voluptate velit esse cillum dolore eu fugiat nulla pariatur. Excepteur sint occaecat cupidatat non proident, sunt in culpa qui officia deserunt mollit anim id est laborum.

\section{Error Analysis}
Lorem ipsum dolor sit amet, consectetur adipiscing elit, sed do eiusmod tempor incididunt ut labore et dolore magna aliqua. Ut enim ad minim veniam, quis nostrud exercitation ullamco laboris nisi ut aliquip ex ea commodo consequat. Duis aute irure dolor in reprehenderit in voluptate velit esse cillum dolore eu fugiat nulla pariatur. Excepteur sint occaecat cupidatat non proident, sunt in culpa qui officia deserunt mollit anim id est laborum.